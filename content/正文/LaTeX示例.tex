\section{LaTeX示例}
\setcounter{figure}{0}
\setcounter{table}{0}

\subsection{图片}

\subsubsection{基本}

单张图片,如\cref{fig:西财Logo单图}。

\begin{figure}[htb]
\centering
\includegraphics[width=0.3\linewidth]{img/swufe_logo.jpg}
\caption{西财Logo}
\label{fig:西财Logo单图}
\end{figure}

\subsubsection{多图并排,共享标题}

多图并排,共享标题,如\cref{fig:西财Logo和小嗷犬}。

\begin{figure}[htb]
\centering
\begin{minipage}{0.5\linewidth}
\centering
\includegraphics[width=0.45\linewidth]{img/swufe_logo.jpg}
\hfill
\includegraphics[width=0.45\linewidth]{img/小嗷犬.png}
\caption{西财Logo和小嗷犬}
\label{fig:西财Logo和小嗷犬}
\end{minipage}
\end{figure}

\subsubsection{多图并排,独立标题}

多图并排,独立标题,如\cref{fig:西财Logo独立标题子图}、\cref{fig:小嗷犬独立标题子图}。

\begin{figure}[htb]
\centering
\begin{minipage}{0.7\linewidth}
\centering
\begin{minipage}{0.49\linewidth}
\centering
\includegraphics[width=0.7\linewidth]{img/swufe_logo.jpg}
\caption{西财Logo}
\label{fig:西财Logo独立标题子图}
\end{minipage}
\hfill
\begin{minipage}{0.49\linewidth}
\centering
\includegraphics[width=0.7\linewidth]{img/小嗷犬.png}
\caption{小嗷犬}
\label{fig:小嗷犬独立标题子图}
\end{minipage}
\end{minipage}
\end{figure}

\subsubsection{子图}

子图,如\cref{fig:西财Logo子图和小嗷犬子图}、\cref{fig:西财Logo子图}、\cref{fig:小嗷犬子图}。

\begin{figure}[htb]
\centering
\begin{minipage}{0.6\linewidth}
\subcaptionbox{西财Logo子图\label{fig:西财Logo子图}}{\includegraphics[width=0.48\linewidth]{img/swufe_logo.jpg}}
\hfill
\subcaptionbox{小嗷犬子图\label{fig:小嗷犬子图}}{\includegraphics[width=0.48\linewidth]{img/小嗷犬.png}}
\caption{西财Logo子图和小嗷犬子图}
\label{fig:西财Logo子图和小嗷犬子图}
\end{minipage}
\end{figure}

\subsection{表格}

\subsubsection{三线表}

三列左对齐,如\cref{tab:三线表1}。

\begin{table}[htb]
\caption{三线表1}
\label{tab:三线表1}
\centering
\renewcommand{\arraystretch}{1}
\begin{tabularx}{\textwidth}{LLL}
\toprule
\textbf{第一列} & \textbf{第二列} & \textbf{第三列} \\
\midrule
A & 1 & 2 \\
B & 3 & 4 \\
\bottomrule
\end{tabularx}
\end{table}

三列右对齐,如\cref{tab:三线表2}。

\begin{table}[htb]
\caption{三线表2}
\label{tab:三线表2}
\centering
\renewcommand{\arraystretch}{1}
\begin{tabularx}{\textwidth}{RRR}
\toprule
\textbf{第一列} & \textbf{第二列} & \textbf{第三列} \\
\midrule
A & 1 & 2 \\
B & 3 & 4 \\
\bottomrule
\end{tabularx}
\end{table}

第一列左对齐,第二列居中对齐,第三列右对齐,如\cref{tab:三线表3}。

\begin{table}[htb]
\caption{三线表3}
\label{tab:三线表3}
\centering
\renewcommand{\arraystretch}{1}
\begin{tabularx}{\textwidth}{LCR}
\toprule
\textbf{第一列} & \textbf{第二列} & \textbf{第三列} \\
\midrule
A & 1 & 2 \\
B & 3 & 4 \\
\bottomrule
\end{tabularx}
\end{table}

\subsection{公式}

\subsubsection{基本}

行内公式:$E=mc^2$。

行间公式:

$$
E=mc^2
$$

\subsubsection{编号}

带编号的公式,如\cref{eq:质能方程}。

\begin{equation}
E=mc^2
\label{eq:质能方程}
\end{equation}

\subsection{代码}

\lstinputlisting[language=Python]{code/bubble_sort.py}

\subsection{算法}

如\cref{alg:冒泡排序算法}。

\begin{algorithm}[htb]
\caption{冒泡排序算法}
\label{alg:冒泡排序算法}
\KwIn{无序数组 $A[1 \dots n]$}
\KwOut{排序后的数组 $A$}
\For{$i = 1$ \KwTo $n-1$}{
    \For{$j = 1$ \KwTo $n-i$}{
        \If{$A[j] > A[j+1]$}{
            交换 $A[j]$ 和 $A[j+1]$
        }
    }
}
\end{algorithm}

\subsection{列表}

\subsubsection{无序列表}

\begin{itemize}
    \item 第一项内容
    \item 第二项内容
\end{itemize}

\subsubsection{有序列表}

\begin{enumerate}
    \item 第一项内容
    \item 第二项内容
\end{enumerate}

\subsubsection{嵌套列表}

\begin{enumerate}
    \item 内容1
    \begin{enumerate}
        \item 内容2
        \item 内容3
    \end{enumerate}
    \item 内容4
\end{enumerate}

\subsection{引用}

我引用西瓜书\cite{西瓜书}。

我引用ResNet\cite{ResNet}。

我引用Transformer\cite{Transformer}。

我引用ResNet和Transformer\cite{ResNet, Transformer}。

我引用西瓜书和Transformer\cite{西瓜书, Transformer}。

我引用西瓜书、ResNet和Transformer\cite{西瓜书, ResNet, Transformer}。