\setcounter{page}{1}
\renewcommand{\abstractname}{摘要}
\begin{abstract}
西南财经大学始于1925年6月3日创建的光华大学,具有悠久的办学历史。抗战爆发后,光华大学于1938年内迁成都,定名为光华大学成都分部,杜甫草堂迤西的这一片地方由此得名光华村。1946年更名为成华大学。

1952至1953年全国院系调整中,以成华大学为基础先后并入西南地区16所财经院校或综合大学的财经系科,组建四川财经学院,是建国之初全国高等院校分区布局中的四所财经学院之一,由国家高教部主管。学校荟萃了我国著名经济学家陈豹隐、李孝同、彭迪先等一批著名教授。1952年10月由西南军政委员会(1953年3月改建为西南行政委员会)领导,1954年1月划归四川省政府领导,1954年12月由中央政府高等教育部直管,1958年7月改由四川省政府主管。1960年分设四川财经学院和四川科学技术学院,1961年合并更名为成都大学,文革期间历尽沧桑,1978年恢复为四川财经学院。1979年划归中国人民银行主管,1985年更名为西南财经大学。2000年以独立建制划转教育部管理。

1997年进入国家“211工程”建设高校,2010年成为国家教育体制改革试点高校,2011年进入国家“985工程”优势学科创新平台建设高校。

\noindent{\textbf{关键词:}西南财经大学;历史沿革;211工程}
\end{abstract}

